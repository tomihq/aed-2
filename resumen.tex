\documentclass[10pt,a4paper]{article}
\usepackage{blindtext}
\usepackage{subcaption}
\usepackage{graphicx}
\usepackage{tikz}
\usepackage{amssymb}
\usepackage{caption}
\usepackage{amsmath}
\usepackage{circuitikz}
\usepackage{hyperref}
\usepackage{amssymb}
\usepackage{amsmath}
\input{AEDmacros}

\title{Algoritmos y Estructuras de Datos II}
\author{Tomás Agustín Hernández}
\date{}

\begin{document}
\maketitle

\begin{figure}[b]
    \centering
    \begin{tikzpicture}[remember picture,overlay]
        \node[anchor=south east, inner sep=0pt, xshift=-1cm, yshift=2cm] at (current page.south east) {
            \begin{minipage}[b]{0.5\textwidth}
                \includegraphics[width=\linewidth]{logo_uba.jpg}
                \label{fig:bottom}
            \end{minipage}
        };
    \end{tikzpicture}
\end{figure}

\newpage
\section{Especificación}
\subsection*{Consideraciones importantes / Reminders}
\begin{itemize}
    \item Utilizar operadores luego: Si estoy en LPO (Lógica de Primer Orden) utilizar los operadores luego si vemos que hay una posible indefinición como una división, o ingresar a una lista a un índice. Recordar que el para todo y un existe, aunque esté acotado por un rango, los cuantificadores predican IGUAL para todos los valores. Entonces, aunque diga que x es positivo, también probará dividir inclusive por 0 y estallará.
    \item Recordar las condiciones bidireccionales
        \begin{itemize}
            \item Si por algún motivo tengo que armar una “lista”, como, por ejemplo, los divisores de un número x tengo que indicar que, si el número divide a x, entonces ese número está en res, pero además todos los valores que están en res DIVIDEN a x. Es una condición bidireccional. 
            \item Otro ejemplo puede ser que tenga que considerar el máximo de una lista, si todos los valores y que están en la lista son menores que res entonces significa que res también pertenece a esa lista original.
        \end{itemize}
    \item Recordar el significado de los cuantificadores con dos variables al mismo tiempo: En la lógica se ejecutan todos de uno a la vez. Es decir, si tengo que poner un para todo adentro de un para todo entonces hago un para todo solo con dos variables y listo.
    \item Recordar que cuando en un procedimiento llamo a un predicado y ese predicado devuelve algo de un para todo, existe (básicamente un valor de verdad) tengo que castear ese valor en el procedimiento porque son dos mundos distintos.
    Ej: asegura: { res = True \(\iff\) predicado}
    \item Los predicados y funciones auxiliares no describen problemas. Son herramientas sintácticas para descomponer predicados.
    \begin{itemize}
        \item Los procedimientos pueden llamar a funciones auxiliares o predicados. Un procedimiento no puede llamar a otro procedimiento.
        \item Los predicados pueden llamar a predicados o auxiliares. 
        \item Las auxiliares solo pueden llamar auxiliares.
    \end{itemize}
    \item No usamos nunca \(==\) en especificación, usamos siempre \(=\) y estamos comparando, no asignando.
    \item No existe el guardar o asignar en el mundo de la lógica. No puedo guardar en una lista en un índice específico porque si un valor. Para esto solemos usar que x valor pertenecerá a esta lista, por ejemplo.
    \item Si tengo un algoritmo que cumple una funcionalidad específica con un require más débil, puedo poner el require más restrictivo y va a funcionar igual pero NO al revés.
\end{itemize}



\end{document}